\section{Aufbau der Arbeit}

\subsection{Begriffsdefinitionen}
\begin{description}
 \item[Test] ist eine, meist automatisierte, Prüfung des Programmverhaltens bei definierten Eingabeparametern
 
 \item[Code-Qualität] beinhaltet die Qualitäten Lesbarkeit, Testbarkeit, Wartbarkeit, Erweiterbarkeit, Geringe Komplexität
 \item[Metrik] Eine Softwaremetrik ist eine statische oder dynamische Codeanalyse zur Generierung von statistischen Informationen über den Source-Code. Beispiele: Testabdeckung, Anzahl Codezeilen, Anzahl Bad Smells pro Codezeile.
 
 \item[Testabdeckung] auch: Testfallabdeckung oder Code-Coverage. Eine dynamische Code-Metrik die angibt, welche Codezeilen durch keinen Test abgedeckt wurde. Es wird unterschieden in die Stufen C0, C1 und C2 mit steigender Komplexität der Messung.\\
 C0: Messung jeder Zeile, ob diese ausgeführt wurde\\
 C1: Messung jedes Zweigs jeder Zeile, ob dieser ausgeführt wurde\\
 C2: Messung jedes möglichen Codepfades, ob dieser ausgeführt wurde
 %TODO Quelle http://blog.abakas.com/2008/04/code-coverage-complexity.html
 \item[Bad Smell] oder Code-Smell. Ist ein Anzeichen für eine suboptimale Code-Stelle, die auch ein Hinweis auf ein größeres Design-Problem sein kann. Oft auch ein Kandidat für ein Refaktoring
 \item[TDD] Test-Driven-Development, bezeichnet den Prozess der Testgetriebenen Entwicklung
 \item[BDD] Behavior-Driven-Development (Verhaltensgetriebene Entwicklung). Prinzipiell eine Umformulierung von TDD zur Unterstützung der Businessprozesse 
 %\item[ATDD] Acceptance-Test-driven-development. Im Gegensatz zu Unittests beim reinen TDD, stehen die Akzeptanztests
\end{description}

