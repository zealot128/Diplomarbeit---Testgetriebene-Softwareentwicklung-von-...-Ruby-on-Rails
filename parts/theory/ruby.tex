\chapter{Die Programmiersprache Ruby}
\label{sec:ruby}
\epigraph{Sometimes people jot down pseudo-code on paper. If that pseudo-code runs directly on their computers, it's best, isn't it? Ruby tries to be like that, like pseudo-code that runs.}{Yukihiro Matsumoto}

Ruby\index{Ruby} ist eine Programmiersprache, die von 1993 von Yukihiro Matsumoto bis heute entwickelt wurde. Dabei ließ er sich von seinen Lieblingsprogrammiersprachen Perl, Smalltalk, Eiffel, Ada und Lisp inspirieren, um eine neue Programmiersprache zu entwickeln, die sowohl funktionale, objektorientierte als auch prozedurale Programmierung ermöglicht \citep{ruby_visual_identity_team_about_2011}.

Eine vollständige Einführung in Ruby\index{Ruby} zu geben würde den Rahmen dieser Diplomarbeit sprengen. Stattdessen legen wir einen Querschnitt durch die Sprache an und stellen die Hauptmerkmale und Unterschiede zu anderen Sprachen heraus. Auch werden Auswirkungen auf das Testen diskutiert und mögliche Testwerkzeuge vorgestellt.

\section{Einführung in Ruby}
\marginline{\includegraphics[width=0.8\marginparwidth]{material/ruby.png}
Ruby ist eine Multiparadigma-Sprache}
Ruby\index{Ruby} ist eine interpretierte Sprache, auch Skriptsprache genannt. Dies bedeutet, dass der Programmcode zur Laufzeit analysiert und ausgeführt wird. Ruby ist auch eine Multiparadigma-Sprache, die Objektorientierung, prozedurale und funktionale Programmierung unterstützt.
\begin{description}
 \item[Prozedural] Funktionen und Variablen können außerhalb von Klassen definiert werden, in dem sogenannten \texttt{main}-Objekt.
 \item[Objektorientierung] Alle Datentypen sind Objekte. Alle Variablen beinhalten Referenzen auf ein Objekt. Dies betrifft auch die primitiven Datentypen wie Integer und String.
 \item[Funktional] Anonyme Funktionen und Closures sind Sprachbestandteil. Alle Statements haben einen Rückgabewert. Innerhalb einer Funktion ist dies immer das letzte Statement, falls kein expliziter Rücksprungpunkt gesetzt wurde.
\end{description}


Das Ziel von Ruby\index{Ruby} ist es, nicht (nur) maschinenlesbar zu sein, sondern vor allem die Lesbarkeit und Nutzbarkeit durch Menschen zu verbessern. Dies drückt sich durch eine Syntax aus, die oft laut als englische Sprache gelesen werden kann und an vielen Stellen auf den Einsatz von Sonderzeichen verzichtet. So ist die Klammerung von Funktionsaufrufen optional und kann weggelassen werden, solange die Zuordnung der Parameter eindeutig ist. Auch hält Ruby eine Vielzahl von redundanten Keywords bereit ("syntaktischer Zucker"), um dem Programmierer mehrere Wege zur Lösung seines Problems zu eröffnen.

\setlength{\epigraphwidth}{\marginparwidth}
\marginline{\epigraph{Ruby is simple in appearance, but is very complex inside, just like our human body}{Yukihiro Matsumoto}}
\setlength{\epigraphwidth}{0.8\textwidth}

Im Nachfolgenden sind einige Beispiele für die Verwendung von Ruby\index{Ruby} dargestellt, insbesondere die "`Alles ist ein Objekt"'-Philosophie. Dazu kommt die \textbf{Interaktive Ruby Shell (IRB)} zum Einsatz, welche im Dialogverfahren Ruby-Code auswertet. Eingaben sind mit \verb+>>+ gekennzeichnet, die Ergebnisse darunter mit \texttt{=>}.

 \begin{ruby}[label=Interaktive Ruby Sitzung (IRB)]
  \PY{g+gp}{>> }\PY{l+m+mi}{2}\PY{o}{.}\PY{n}{even?}
  \PY{g+go}{=> true}
  \PY{g+gp}{>> }\PY{l+s+s2}{"}\PY{l+s+s2}{hallo}\PY{l+s+s2}{"}\PY{o}{.}\PY{n}{upcase}
  \PY{g+go}{=> "HALLO"}
  \PY{g+gp}{>> }\PY{n+no}{Date}\PY{o}{.}\PY{n}{today} \PY{o}{+} \PY{l+m+mi}{2}
  \PY{g+go}{=> #<Date: 2011-06-30>}
  \PY{g+gp}{>> }\PY{n}{a} \PY{o}{=} \PY{l+m+mi}{4} \PY{o}{+} \PY{n+no}{Math}\PY{o}{.}\PY{n}{sqrt}\PY{p}{(}\PY{l+m+mi}{9}\PY{p}{)}
  \PY{g+go}{=> 7.0}
  \PY{g+gp}{>> }\PY{k}{if} \PY{p}{(}\PY{l+m+mi}{0}\PY{o}{.}\PY{n}{.}\PY{l+m+mi}{10}\PY{p}{)}\PY{o}{.}\PY{n}{include?} \PY{n}{a}
  \PY{g+gp}{>> }  \PY{n+nb}{puts} \PY{l+s+s2}{"}\PY{l+s+s2}{a liegt zwischen 0 und 10}\PY{l+s+s2}{"}
  \PY{g+gp}{>> }\PY{k}{end}
  \PY{g+go}{=> a liegt zwischen 0 und 10}
 \end{ruby}
\codecaption{Ruby Beispiele}

In den ersten beiden Beispielen sieht man, dass Integer und Strings Objekte sind und über Instanzmethoden verfügen. Im ersten Beispiel wird geprüft, ob die Zahl gerade ist. Dabei existiert eine Konvention, dass bool'sche Methoden mit einem Fragezeichen am Ende notiert werden. Im dritten Beispiel wird eine Klassenmethode \texttt{today} auf die Klasse \texttt{Date} ausgeführt, welche ein Datumsobjekt konstruiert und zurückliefert. Da auch die Nutzung von Operatoren letztendlich nur syntaktischer Zucker für Methodenaufrufe sind, wird die Instanz-Methode \texttt{.+()} auf dieses Datums-Objekt ausgeführt und liefert ein neues Datumsobjekt, welches zwei Tage in der Zukunft liegt.\\
Im vierten Beispiel wird der Einsatz von Variablen demonstriert.
Das letzte Beispiel zeigt den Einsatz von Kontrollstrukturen. Als Besonderheit sei hier auf den Ausdruck vom Typ \texttt{Range} \texttt{(0..10)} hingewiesen, der ein Intervall für den Integerzahlenbereich von 0 bis einschließlich 10 liefert. Die Methode \texttt{.include?(a)} testet nun, ob die Variable \texttt{a} in diesem Intervall liegt. Bei Eindeutigkeit können, wie oben bereits angesprochen, die Klammern eines Methodenaufrufes weggelassen werden.

Weiterhin erlaubt Ruby\index{Ruby} die Arbeit mit Lambdas, also anonymen Funktionen. Eine beliebte Verwendungsmöglichkeit ist die Bearbeitung von Arrays und listenähnlichen Strukturen.

% SNIPPET: [language=Ruby,label=Ruby Beispiel: Lambdas,caption=Ruby Beispiel: Lambdas]
% >> adder = lambda { |a,b|  a + b }
% >> adder.call(1,2)
% => 3
%
% # Sortiere nach Standardvergleichsoperator
% >> [4,5,7,3].sort()
% => [3, 4, 5, 7]
%
% # Es kann auch eine benutzerdefinierte Sortierfunktion
% # angegeben werden
% >> [ "string",  "rails",  "ruby" ].sort_by{ |item| item.length }
% => ["ruby", "rails", "string"]
%
% # Die Quadratzahlen von 1 bis 5
% >> (1..5).map{|element| element * 2}
% => [2, 4, 6, 8, 10]
%

\begin{ruby}[label=IRB]
\PY{g+gp}{>> }\PY{n}{adder} \PY{o}{=} \PY{n+nb}{lambda} \PY{p}{\PYZob{}} \PY{o}{|}\PY{n}{a}\PY{p}{,}\PY{n}{b}\PY{o}{|}  \PY{n}{a} \PY{o}{+} \PY{n}{b} \PY{p}{\PYZcb{}}
\PY{g+gp}{>> }\PY{n}{adder}\PY{o}{.}\PY{n}{call}\PY{p}{(}\PY{l+m+mi}{1}\PY{p}{,}\PY{l+m+mi}{2}\PY{p}{)}
\PY{g+go}{=> 3}

\PY{g+go}{# Sortiere nach Standardvergleichsoperator}
\PY{g+gp}{>> }\PY{o}{[}\PY{l+m+mi}{4}\PY{p}{,}\PY{l+m+mi}{5}\PY{p}{,}\PY{l+m+mi}{7}\PY{p}{,}\PY{l+m+mi}{3}\PY{o}{]}\PY{o}{.}\PY{n}{sort}\PY{p}{(}\PY{p}{)}
\PY{g+go}{=> [3, 4, 5, 7]}

\PY{g+go}{# Es kann auch eine benutzerdefinierte Sortierfunktion}
\PY{g+go}{# angegeben werden}
\PY{g+gp}{>> }\PY{o}{[} \PY{l+s+s2}{"}\PY{l+s+s2}{string}\PY{l+s+s2}{"}\PY{p}{,}  \PY{l+s+s2}{"}\PY{l+s+s2}{rails}\PY{l+s+s2}{"}\PY{p}{,}  \PY{l+s+s2}{"}\PY{l+s+s2}{ruby}\PY{l+s+s2}{"} \PY{o}{]}\PY{o}{.}\PY{n}{sort\PYZus{}by}\PY{p}{\PYZob{}} \PY{o}{|}\PY{n}{item}\PY{o}{|} \PY{n}{item}\PY{o}{.}\PY{n}{length} \PY{p}{\PYZcb{}}
\PY{g+go}{=> ["ruby", "rails", "string"]}

\PY{g+go}{# Die Quadratzahlen von 1 bis 5}
\PY{g+gp}{>> }\PY{p}{(}\PY{l+m+mi}{1}\PY{o}{.}\PY{n}{.}\PY{l+m+mi}{5}\PY{p}{)}\PY{o}{.}\PY{n}{map}\PY{p}{\PYZob{}}\PY{o}{|}\PY{n}{element}\PY{o}{|} \PY{n}{element} \PY{o}{*} \PY{l+m+mi}{2}\PY{p}{\PYZcb{}}
\PY{g+go}{=> [2, 4, 6, 8, 10]}
\end{ruby}
\codecaption{Ruby Beispiel: Blöcke}
Das erste Beispiel zeigt, wie Funktionsausdrücke in Variablen gespeichert werden können, um später aufgerufen zu werden. Hier wird eine Addierfunktion definiert und aufgerufen.\\
Das zweite Beispiel zeigt, wie Arrays verwendet werden und durch eine bereits eingebaute Methode \texttt{sort} sortiert werden können. Falls die enthaltenen Objekte nicht trivial verglichen werden können, ermöglicht die Methode \texttt{sort\_by} des dritten Beispiels die Angabe einer benutzerdefinierten Sortierfunktion, hier z.B. die Sortierung nach der Länge eines Strings.\\
Im letzten Beispiel wird demonstriert, wie die Funktion \texttt{map} verwendet wird, die eine beliebige Funktion auf alle Elemente der Liste ausführt. Hier quadrieren wir alle Elemente unserer Liste und erhalten die quadrierte Liste als Rückgabewert von \texttt{map} (Die originale Liste bleibt dabei unverändert).



\paragraph{Typ- und Objektsystem}
Wie schon erwähnt, sind bei Ruby\index{Ruby} alle Datentypen ein Objekt. Dies schließt insbesondere Klassen und primitive Datentypen mit ein, wie wir im folgenden Beispiel sehen können:
% SNIPPET:
% >> 2.class
% => Fixnum
% >> Fixnum.class
% => Class
% >> Class.class
% => Class
%
% >> Fixnum.superclass
% => Integer
% >> Fixnum.ancestors
% => [Fixnum, Integer, Precision, Numeric, Comparable, Object, Kernel]
%
\begin{ruby}[label=IRB]
\PY{g+gp}{>> }\PY{l+m+mi}{2}\PY{o}{.}\PY{n}{class}
\PY{g+go}{=> Fixnum}
\PY{g+gp}{>> }\PY{n+no}{Fixnum}\PY{o}{.}\PY{n}{class}
\PY{g+go}{=> Class}
\PY{g+gp}{>> }\PY{n+no}{Class}\PY{o}{.}\PY{n}{class}
\PY{g+go}{=> Class}

\PY{g+gp}{>> }\PY{n+no}{Fixnum}\PY{o}{.}\PY{n}{superclass}
\PY{g+go}{=> Integer}
\PY{g+gp}{>> }\PY{n+no}{Fixnum}\PY{o}{.}\PY{n}{ancestors}
\PY{g+go}{=> [Fixnum, Integer, Precision, Numeric, Comparable, Object, Kernel]}
\end{ruby}
\codecaption{Klassenhierarchien}
% label{fig:f37300}

Das Literal \texttt{2} ist somit ein Objekt vom Typ \texttt{Fixnum}. Die Klasse Fixnum ist ihrerseits vom Typ \texttt{Class}. Da Ruby\index{Ruby} sowohl (Einfach-)Ableitungen als auch sogenannte Includes bzw. Mixins unterstützt, kann eine Klasse auch eine Menge von "`Vorfahren"' haben. Die gezeigte Klasse \texttt{Fixnum} verfügt somit standardmäßig sogar über 7 Oberklassen.


Ruby\index{Ruby} ist \textbf{dynamisch stark} typisiert. \textit{Dynamisch} bedeutet, dass die Prüfung des Typs einer Variable zur Laufzeit stattfindet und sich dieser Typ ausschließlich aus dem aktuell beinhalteten Objekt ergibt. Durch die \textit{starke} Typisierung ist es aber nicht möglich, invalide Operationen auf Typ-inkompatible Objekte auszuführen, beispielsweise eine Addition von Integer mit String. \\
\borderquote{"When I see a bird that walks like a duck and swims like a duck and quacks like a duck, I call that bird a duck."}{James Whitcomb Riley}
Rubys Typsystem ist "`Duck-typed"', d.h. dass die Semantiken eines Objekts nicht durch seine Klasse und Ableitungshierarchie, sondern durch seine Methoden und Attribute bestimmt wird.

Ruby\index{Ruby} verfügt über lexikalische und dynamische Bindung\footnote{
\textbf{Static Scoping}: Variablen werden zur Compilezeit gebunden, ohne den aufrufenden Code zu berücksichtigen.\\
\textbf{Dynamic Scoping}: Variablen-Bindung kann nur im Moment der Ausführung des Codes festgestellt werden.}, letztere wird allerdings seltener verwendet. In der Basissyntax verwendet Ruby\index{Ruby} statische Bindung. Es existiert eine im Ruby-Core enthaltene Bibliothek \texttt{Dynamic} zum dynamischen Binden, falls dies gewünscht sein sollte.


\paragraph{Reflektion und Introspection} Sprachen mit Reflektion erlauben es, ihre Strukturen zur Laufzeit zu analysieren und zu verändern. So können in Ruby\index{Ruby} z.B. Objekte Informationen über ihre vorhandenen Instanzvariablen oder Methoden abfragen. Wichtig anzumerken sei noch, dass Klassen in Ruby nie geschlossen sind, sondern jederzeit erweitert werden können und vorhandene Methoden überschrieben werden können. So ist es z.B. möglich, die String-Klasse um eigene Funktionen zu erweitern.

% SNIPPET: [language=Ruby,label=Ruby Beispiel offene Klassen,caption=Ruby Beispiel offene Klassen]
% >> class String
% >>   def remove_whitespace
% >>     self.gsub(/\s+/, "")
% >>   end
% >> end
%
% >> "Dies ist ein Test".remove_whitespace
% => "DiesisteinTest"
%
%
\begin{ruby}[label=IRB]
\PY{g+gp}{>> }\PY{k}{class} \PY{n+nc}{String}
\PY{g+gp}{>> }  \PY{k}{def} \PY{n+nf}{remove\PYZus{}whitespace}
\PY{g+gp}{>> }    \PY{n+nb}{self}\PY{o}{.}\PY{n}{gsub}\PY{p}{(}\PY{l+s+sr}{/}\PY{l+s+sr}{\PYZbs{}}\PY{l+s+sr}{s+}\PY{l+s+sr}{/}\PY{p}{,} \PY{l+s+s2}{"}\PY{l+s+s2}{"}\PY{p}{)}
\PY{g+gp}{>> }  \PY{k}{end}
\PY{g+gp}{>> }\PY{k}{end}

\PY{g+gp}{>> }\PY{l+s+s2}{"}\PY{l+s+s2}{Dies ist ein Test}\PY{l+s+s2}{"}\PY{o}{.}\PY{n}{remove\PYZus{}whitespace}
\PY{g+go}{=> "DiesisteinTest"}
\end{ruby}
\codecaption{Ruby Beispiel offene Klassen}
% label{fig:9bf48b}

\paragraph{Generische Programmierung und Aspekte der Metaprogrammierung}
Metaprogrammierung umfasst die Analyse, Transformation und Generierung von Objektprogrammen durch Metaprogramme \citep{herrmann_2005}.  Sie ermöglicht es, Probleme effektiv zu lösen, die andernfalls nur mit erheblichem Aufwand oder gar nicht zu lösen sind.\\
Ein beliebtes Idiom innerhalb der Ruby-Community ist es, verwendete Methoden auf Basis des Methodennamens zur Laufzeit zu erstellen (Generierung). Dies verwendet z.B. das beliebte ORM\footnote{\glossar{ORM}}-Datenbank\index{Datenbank}framework ActiveRecord\index{ActiveRecord}, um einfache SQL-Statements anhand des Funktionsnamens zu erstellen (Ruby on Rails\index{Ruby-on-Rails} verwendet standardmäßig ActiveRecord als Schnittstelle zur Datenbank).
% SNIPPET:
% >> Person.find_by_first_name("Stefan")
% #  Person Load (0.2ms)  SELECT persons.* FROM persons
% #    WHERE users.first_name = 'Stefan' LIMIT 1
\begin{ruby}[label=IRB]
\PY{g+gp}{>> }\PY{n+no}{Person}\PY{o}{.}\PY{n}{find\PYZus{}by\PYZus{}first\PYZus{}name}\PY{p}{(}\PY{l+s+s2}{"}\PY{l+s+s2}{Stefan}\PY{l+s+s2}{"}\PY{p}{)}
\PY{g+go}{#  Person Load (0.2ms)  SELECT persons.* FROM persons}
\PY{g+go}{#    WHERE users.first\PYZus{}name = 'Stefan' LIMIT 1}
\end{ruby}
\codecaption{Demonstration von generischen ==> VON GENERISCHEN WAS??? METHODEN?}
% label{fig:d0ee1e}


Die Methode \texttt{find\_by\_first\_name} existiert nicht und wird zur Laufzeit auf Basis des Namens gebaut. Dies ist möglich, da Ruby\index{Ruby} sogennante Hooks (Callbacks) bereitstellt. Der Hook \texttt{method\_missing} z.B. wird immer dann aufgerufen, wenn eine nicht-existierende Funktion aufgerufen wurde (wie in unserem Beispiel \texttt{find\_by\_first\_name}). Hier können wir nun die neue Methode auf Basis des gewünschten Funktionsnamens zur Laufzeit erstellen oder andernfalls selbst eine Exception werfen. Ein weiterer Hook ist z.B. auch \texttt{method\_added}, der aufgerufen wird, wenn in einer Klasse eine Methode definiert wird\footnote{Einen guten Überblick über die Callbacks, die Ruby bereitstellt, und was man damit tun kann, finden sie hier: \url{http://www.khelll.com/blog/ruby/ruby-callbacks/}}. Auf diese Weise sind z.B. die Modifikatoren \texttt{private} und \texttt{public} implementiert.

Ein weiteres Beispiel ist die Definition der relationalen Beziehungen zwischen den einzelnen Modell\index{Ruby-on-Rails!Modell}en innerhalb von \glossar{rails} (ebenfalls unter der Verwendung von ActiveRecord\index{ActiveRecord}).
\begin{ruby}[label=app/models/job.rb]
\PY{k}{class} \PY{n+nc}{Job} \PY{o}{<} \PY{n+no}{ActiveRecord}\PY{o}{::}\PY{n+no}{Base}
  \PY{n}{belongs\PYZus{}to} \PY{l+s+ss}{:user}
\PY{k}{end}
\end{ruby}
\codecaption{Nutzung von Metaprogrammierung zur Erstellung von Objektbeziehungen}
% label{fig:a7720a}

Hiermit definieren wir, dass ein Job einem User gehört, es also eine 1:N (oder 1:1)-Beziehung zwischen beiden gibt. Die dafür benötigten Getter und Setter werden mittels des Methodenaufrufs \texttt{belongs\_to} in die Klasse Job geschrieben.

Diese Beispiele sollten als kurzer Einstieg in Ruby\index{Ruby} dienen und einen Querschnitt durch die Besonderheiten der Sprache aufzuzeigen.
Für eine weitere Vertiefung sei das Buch "`Programming Ruby 1.9"' empfohlen, das im Detail auf die neuste Version der Programmiersprache eingeht \citep{hunt_programming_2009}.



\section{Diskussion}

% Typunsicherheit, Geschwindigkeitsnachteil

Dynamisch-typisierte Sprachen, wie Ruby\index{Ruby}, haben gegenüber statisch-typisierten Sprachen einige Nachteile.\randbem{Dynamische Typisierung und Performanz} Oft wird der Geschwindigkeitsnachteil angesprochen, den der Prozess des Interpretierens und die dynamische Typisierung verursachen.
Der genaue Faktor variiert allerdings je nach Algorithmus und Implementierung, stark. Ein beliebter Benchmark, "`shootout.alioth"', vergleicht beliebte Algorithmen der Informatik, implementiert in verschiedenen Sprachen, miteinander. So ergibt sich z.B. in der Gegenüberstellung von Ruby\index{Ruby} mit C ein 4-fach bis 300-fach langsamere Ausführungszeit. Dem gegenüber steht allerdings nur die Hälfte bis $1/7$ der Menge an benötigtem Code \citep{computer_language_benchmarks_game_ruby_2011}. \\
Wichtig ist auch die verwendete Laufzeitumgebung. Neben der Referenzimplementierung von Matsumotu (Ruby MRI), existieren noch JRuby, eine Implementierung auf Basis der Java Virtuellen Maschine und Rubinius. Die letzten beiden unterstützen auch eine sogenannte Just-in-time (JIT)-Kompilierung zur Verbesserung der Performanz bei längerer Ausführungszeit des Programms. Desweiteren gibt es gerade im Bereich Laufzeitoptimierung viel Bewegung innerhalb der Ruby-Implementierungen und fast alle Ruby-Implementierungen nehmen stetig an Geschwindigkeit zu \citep{antonio_cangiano_great_2010}.


Allerdings\randbem{... und fehlende Typsicherheit} bleiben Fehler, die der Compiler bereits entdeckt hätte, bis zur Ausführung oder schlimmstenfalls noch länger unentdeckt. Dazu gehören z.B. Tippfehler, bei denen der Wert einer nicht deklarierten Variable ausgelesen wird. Im Gegensatz zu z.B. PHP, wirft Ruby\index{Ruby} aber dann eine Exception.\\
Auf das Testen hat dies eine direkte Auswirkung. Viele Meinungen belegen, dass eine dynamisch typisierte Sprache mehr Tests benötigt, als eine statisch typisierte \citep{daniel_spiewak_dynamic_2010}.



Ein Vorteil des Interpretierens,\randbem{Vorteile aus der dynamischen Typisierung} also der Übersetzung zur Laufzeit, ist eine hohe Plattformunabhängigkeit und ein leichterer Buildprozess, da das Kompilieren entfällt.
Verfechter dynamischer Sprachen erklären weiterhin, dass diese sich ideal für prototypische Implementierungen eignen, da sich Anforderungen ständig ändern können. Weiterhin hätten Programme dynamischer Sprache eine potenziell hohe Wiederverwendbarkeit und eine höhere Lesbarkeit \citep{meijer_static_2005} \citep{ousterhout_scripting:_1998}.


%
% All diese Methoden können, richtig angewendet, zur Verbesserung der Lesbarkeit der Programme und damit zur Erhöhrung der Wartbarkeit, führen.
%
% Auch das sehr beliebte Testframework Rspec, verwendet Metaprogrammierung, um Testfälle und Zusicherungen wie fast in der englischen Sprache zu formulieren. Dazu werden sämtliche Objekte von Ruby um Funktionen erweitert. Dies ist möglich, da die Klasse "`Objekt"', die Basisklasse (fast) aller Ruby-Klasse ist, um diese Methoden erweitert wurde.
%





\paragraph{Schlussfolgerung}
Die Verwendung von Ruby\index{Ruby} und anderen dynamischen Sprachen birgt durchaus Risiken, die zu beachten sind. Falls man sich dieser Risiken bewusst ist und die Möglichkeiten der Sprache nutzt, um die Lesbarkeit zu verbessern, sind sie gerechtfertigt. Gerade bei der Entwicklung mit kleinen Entwicklerteams und Projekten mit engem Budget können dynamische Sprachen ihre Vorteile ausspielen, da sie eine schnellere Entwicklung ermöglichen. Im Gegensatz zu den meisten auf Syntax von C basierenden Sprachen (z.B. Java oder C\#), ist die Syntax von Ruby äußerst leserlich, da nur wenige Sonderzeichen verwendet werden. Auch ist Ruby sehr ausdrucksstark, weil die Deklaration entfällt, es viel sogenannten syntaktischen Zucker gibt und das flexible Objektsystem viele Möglichkeiten eröffnet. Alles dies kann, richtig angewendet, der Lesbarkeit zuträglich sein.


