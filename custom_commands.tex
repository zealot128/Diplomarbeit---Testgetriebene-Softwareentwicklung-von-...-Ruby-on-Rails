% FullRef:   siehe \ref{sec:bla}   -> siehe 7 - "Kapitelname"
\newcommand{\fullref}[1]{\ref{#1} -- "`\textit{\nameref{#1}}"'}

% Shadowbox
\xdefinecolor{B4}{cmyk}{.25,.05.5,0,0}
\xdefinecolor{B4}{rgb}{0.9,0.9,1}
\xdefinecolor{S10}{cmyk}{0 ,0, 0,.1}
\usepackage{tikz}
\usetikzlibrary{shadows}
\newcommand{\shadebox}[1]{%
\vspace{1\baselineskip}%
\noindent%
\begin{tikzpicture}%
\node at (0,0) [
   anchor=north west,
    rounded corners,
   minimum width={\linewidth-1ex},
   draw=B4,
   fill=B4
  ]
{\parbox{0.9\linewidth}{#1}};
\end{tikzpicture}\vspace{1\baselineskip}}


%%%%%%%%%%%%%%%%%
% Makros fuer den Einsatz von Dingens am Rand
%%%%%%%%%%%%%%%%%
\newcommand{\randbem}[1]{
  \marginpar[\fontsize{9}{10}\selectfont \raggedleft{#1}]
  {\fontsize{9}{10}\selectfont \raggedright{#1}}
}

\newcommand{\bordergraphic}[1]{
  \marginline{\includegraphics[width=0.8\marginparwidth]{#1}}}
\newcommand{\tddred}{\bordergraphic{diagrams/red.png}}
\newcommand{\tddgreen}{\bordergraphic{diagrams/green.png}}
\newcommand{\tddrefactor}{\bordergraphic{diagrams/refactor.png}}

\newcommand{\borderquote}[2]{
\setlength{\epigraphwidth}{\marginparwidth}
\marginpar{\fontsize{9}{10} \epigraph{#1}{#2}}
\setlength{\epigraphwidth}{0.8\textwidth}
}


\newcommand{\cbox}{\includegraphics[width=0.2cm]{material/cbox.png} \ }
% Einheitliche Bildquelle als zusaetzliches Label
\newcommand{\imgsource}[1]{\captionsetup{font={footnotesize,bf,it}} \caption*{#1}}

% %%% Tiefe des Inhaltsverzeichnis  beschraenken
% \setcounter{tocdepth}{2}	%kleineres TOC

% Langer Pfeil fuer den Fließtext
\newcommand{\arr}{$ \Longrightarrow$}


% Standard-Listings fuer Ruby laden
\lstloadlanguages{Ruby}
\lstset{%
  basicstyle=\ttfamily\footnotesize\color{black},
  commentstyle = \ttfamily\color{gray},
  keywordstyle=\ttfamily\color{blue},
  stringstyle=\color{red},
  showspaces=false,               % show spaces adding particular underscores
  showstringspaces=false,
  frame=single,
  breaklines=true
}
% Umlaute in Listings ermoeglichen
\lstset{literate=%
{Ö}{{\"O}}1
{Ä}{{\"A}}1
{Ü}{{\"U}}1
{ß}{{\ss}}2
{ü}{{\"u}}1
{ä}{{\"a}}1
{ö}{{\"o}}1
}


% \newcommand{\borderquote}[2]{
% \marginnote{#1}
% }

%%%%%%%%%%%%
% Makros fuer Hihglighting mittels Pygments
\usepackage{fancyvrb}
% brauchbar:
%             py_bw         <-- Black White, sehr brauchbar fuer druck, nur kursiv und bold
%             py_colorful   <kraeftige Farben ausser Kommentare...
%
% maessig:    py_manni
%             py_borland
\input{styles/py_tango}


\DefineVerbatimEnvironment{ruby}{Verbatim}{numbers=left,frame=single,stepnumber=1,numbersep=1pt,fontsize=\relsize{-1},
commandchars=\\\{\}}


\newcommand{\codecaption}[1]{
\captionsetup{type=lstlisting,position=top}
\caption{#1}
}



